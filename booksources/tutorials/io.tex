% -*- latex -*-
%%%%%%%%%%%%%%%%%%%%%%%%%%%%%%%%%%%%%%%%%%%%%%%%%%%%%%%%%%%%%%%%
%%%%%%%%%%%%%%%%%%%%%%%%%%%%%%%%%%%%%%%%%%%%%%%%%%%%%%%%%%%%%%%%
%%%%
%%%% This text file is part of the source of 
%%%% `Parallel Programming in MPI and OpenMP'
%%%% by Victor Eijkhout, copyright 2012-2021
%%%%
%%%% io.tex : parallel I/O
%%%%
%%%%%%%%%%%%%%%%%%%%%%%%%%%%%%%%%%%%%%%%%%%%%%%%%%%%%%%%%%%%%%%%
%%%%%%%%%%%%%%%%%%%%%%%%%%%%%%%%%%%%%%%%%%%%%%%%%%%%%%%%%%%%%%%%

For a great discussion see~\cite{Mendez:ParallelIOpage},
from which figures here are taken.

\Level 0 {Use sequential I/O}

MPI processes can do anything a regular process can,
including opening a file.
This is the simplest form of parallel I/O:
every MPI process opens its own file.
To prevent write collisions,
\begin{itemize}
\item you use \indexmpishow{MPI_Comm_rank} to generate a unique file name, or
\item you use a local file system, typically \n{/tmp}, that is unique
  per process, or at least per the group of processes on a node.
\end{itemize}

For reading it is actually possible for all processes to open the same file,
but for reading this is not really feasible. Hence the unique files.

\Level 0 {MPI I/O}

In chapter~\ref{ch:mpi-io} we discussed MPI I/O.
This is a way for all processes on a communicator to open a single file,
and write to it in a coordinated fashion.
This has the big advantage that the end result is an ordinary Unix file.

\Level 0 {Higher level libraries}

Libraries such as \indexterm{NetCDF} or \indexterm{HDF5}
offer advantages over MPI I/O:
\begin{itemize}
\item Files can be OS-independent, removing worries such
  as about \indexterm{little-endian} storage.
\item Files are self-documenting: they contain the metadata describing their contents.
\end{itemize}


