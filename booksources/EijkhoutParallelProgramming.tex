% -*- latex -*-
%%%%%%%%%%%%%%%%%%%%%%%%%%%%%%%%%%%%%%%%%%%%%%%%%%%%%%%%%%%%%%%%
%%%%%%%%%%%%%%%%%%%%%%%%%%%%%%%%%%%%%%%%%%%%%%%%%%%%%%%%%%%%%%%%
%%%%
%%%% This text file is part of the source of 
%%%% `Parallel Computing'
%%%% by Victor Eijkhout, copyright 2012-2021
%%%%
%%%% parcompbook.tex : master file for the book
%%%%
%%%%%%%%%%%%%%%%%%%%%%%%%%%%%%%%%%%%%%%%%%%%%%%%%%%%%%%%%%%%%%%%
%%%%%%%%%%%%%%%%%%%%%%%%%%%%%%%%%%%%%%%%%%%%%%%%%%%%%%%%%%%%%%%%

\documentclass[11pt,letterpaper,twoside,openany]{boek3}
%\documentclass{book}

\usepackage{verbatim}

\usepackage{comment}
\specialcomment{tacc}{\def\CommentCutFile{tacc.cut}}{}
\newif\ifIncludeAnswers
\IncludeAnswersfalse
\input inex
\includecomment{gpu}
\includecomment{review}
\includecomment{book}

\usepackage{scrwfile}
\usepackage{graphicx,outliner}
\usepackage{wrapfig}
\usepackage{pgfplots}  
\usepgfplotslibrary{external}
\tikzexternalize
\tikzsetexternalprefix{plots/}
\pgfplotsset{width=5in,compat=1.7}  

% fancy text stuff
\usepackage{fontspec}
\setmainfont[
  Extension=.otf,
  UprightFont={*-Regular},
  BoldFont={*-Bold},
  ItalicFont={*-Italic},
  BoldItalicFont={*-BoldItalic}
]{LibertinusSerif}
\usepackage{unicode-math}
\setmathfont{LibertinusMath-Regular.otf}
%%%%%%%%%%%%%%%%%%%
\usepackage{dirtree} % ,times
% table stuff
\usepackage{booktabs,multicol,multirow}

% AMS math
%\usepackage[fleqn]{amsmath}
%\usepackage{amssymb}

% custom arrays and tables
\usepackage{array} %,multirow,multicol}
\newcolumntype{R}{>{\hbox to 1.2em\bgroup\hss}{r}<{\egroup}}
\newcolumntype{T}{>{\hbox to 8em\bgroup}{c}<{\hss\egroup}}

% algorithms
\usepackage[algo2e,noline,noend]{algorithm2e}
\newenvironment{displayalgorithm}
 {\begin{algorithm2e}[H]\leftskip=\unitindent \DontPrintSemicolon
  \SetKwInOut{Input}{Input}\SetKwInOut{Output}{Output}
 }
 {\end{algorithm2e}}
\newenvironment{displayprocedure}[2]
 {\everymath{\strut}
  \begin{procedure}[H]\leftskip=\unitindent\caption{#1(#2)}}
 {\end{procedure}}

\def\svnrev{428}
% dashed lines; this may interfere with other table packages
\usepackage{arydshln}

\edef\revision{\svnrev}
\def\lulurevision{}

%
% page layout
%
\usepackage{geometry}
\addtolength{\textwidth}{.5in}
\addtolength{\textheight}{.5in}
\addtolength{\evensidemargin}{-.5in}
\renewcommand\topfraction{.95}
\renewcommand\floatpagefraction{.75}

\usepackage{fancyhdr}
\pagestyle{fancy}\fancyhead{}\fancyfoot{}
% remove uppercase from fancy defs
\makeatletter
\def\chaptermark#1{\markboth {{\ifnum \c@secnumdepth>\m@ne
 \thechapter. \ \fi #1}}{}}
\def\sectionmark#1{\markright{{\ifnum \c@secnumdepth >\z@
 \thesection. \ \fi #1}}}
\makeatother
% now the fancy specs
%\fancyhead[LE]{\thepage \hskip.5\unitindent/\hskip.5\unitindent \leftmark}
%\fancyhead[RO]{\rightmark \hskip.5\unitindent/\hskip.5\unitindent \thepage}
\fancyhead[LE]{\leftmark}
\fancyfoot[LE]{\thepage}
\fancyhead[RO]{\rightmark}
\fancyfoot[RO]{\thepage}
\fancyfoot[RE]{\footnotesize\sl Parallel Computing -- r\revision}
\begin{lulu}
\fancyfoot[RE]{\footnotesize\sl Parallel Computing}
\end{lulu}
\fancyfoot[LO]{\footnotesize\sl Victor Eijkhout}

\newwrite\nx
\newcommand\CHAPTER[2]{
  \Level 0 {#1}\label{ch:#2}
  \def\chapshortname{#2}

  % start scooping up example codes used in this chapter
  \addchaptersource{header}%kludge because we have a bug for zero sources

  % input the chapter
  \SetBaseLevel 1 \input chapters/#2

  % include the sources used in this chapter
  \newpage
  \listchaptersources\chapshortname

  % write this chapter to a list of chapters
  \write\chapterlist{\chapshortname}
  \openout\nx=exercises/\chapshortname-nx.tex
  \write\nx{\arabic{excounter}}
  \closeout\nx

  \SetBaseLevel 0
}

\includecomment{tutorials}
\newcommand\TUTORIAL[2]{
  \Level 0 {#1}\label{tut:#2}
  \def\chapshortname{#2}
  \setcounter{excounter}0\relax
  \SetBaseLevel 1 \input tutorials/#2
  \write\chapterlist{\chapshortname}
  \openout\nx=exercises/\chapshortname-nx.tex
  \write\nx{\arabic{excounter}}
  \closeout\nx
  \SetBaseLevel 0
}
\newif\ifprojects\projectsfalse
\newcommand\PROJECT[2]{
\ifprojects \vfill\pagebreak \else \projectstrue \fi
\Level 0 {#1}\label{prj:#2}
\def\chapshortname{#2}
{\SetBaseLevel 1 \input projects/#2
\write\chapterlist{\chapshortname}
\openout\nx=exercises/\chapshortname-nx.tex
\write\nx{\arabic{excounter}}
\closeout\nx
\SetBaseLevel 0
}}
\newcommand\APPENDIX[3]{
  \vfill\pagebreak \Level 1 {#1}\label{app:#3}
  \def\chapshortname{#3}
  {\SetBaseLevel 2 {\index{#2|(}}
   \setcounter{excounter}0
   \input appendices/#3 {\index{#2|)}}
   \write\chapterlist{\chapshortname}
   \openout\nx=exercises/\chapshortname-nx.tex
   \write\nx{\arabic{excounter}}
   \closeout\nx
   \SetBaseLevel 0
}}
\newcommand\APPENDIXac[3]{
  \vfill\pagebreak \Level 1 {#1}\label{app:#3}
  \def\chapshortname{#3}
  {\SetBaseLevel 2 {\indexacstart{#2}}
   \setcounter{excounter}0
   \input appendices/#3 {\indexacend{#2}}
   \write\chapterlist{\chapshortname}
   \openout\nx=exercises/\chapshortname-nx.tex
   \write\nx{\arabic{excounter}}
   \closeout\nx
   \SetBaseLevel 0
}}

\newcommand\maillink[3]{
  \href{mailto:eijkhout@tacc.utexas.edu?subject=comment on section #1 "#2"}
    {comments on this #3?}\par
}
\renewcommand\maillink[3]{}

\input commonmacs
\input acromacs
\input bookmacs
\input exmacs
\input tutmacs
\usepackage[original]{imakeidx}
\input idxmacs
\input listingmacs
\input snippetmacs

\usepackage[xetex,colorlinks]{hyperref}
\hypersetup{bookmarksopen=true}
\usepackage{etoolbox}
%%
%% refer to sections in the HPC book
%%
\usepackage{xr-hyper}
\externaldocument[HPSC-]{../scientific-computing-private/scicompbook}
\newcommand\HPSCref[1]{HPSC-\nobreak\ref{HPSC-#1}}
\usepackage[all]{hypcap}

\def\publicdraft{{\bf\normalsize \relax Public draft - open for comments}}
\def\revdate{2nd edition 2020, formatted \today}
\begin{lulu}
\def\publicdraft{}
\def\revdate{2nd edition 2020}
\end{lulu}

\newwrite\chapterlist \openout\chapterlist=chapternames.tex

\begin{document}

\author{Victor Eijkhout}
\title{Parallel Programming for Science and Engineering\\
\small Using MPI, OpenMP, and the PETSc library}
\expandafter\date\expandafter{\revdate}
\maketitle
\publicdraft
\vfill\pagebreak 

\input copyright
\vfill\pagebreak 
\input introduction
\vfill\pagebreak 

{\setcounter{tocdepth}{1}
\tableofcontents
\setcounter{tocdepth}{2}
}

\acresetall
\part{MPI}

\input chapters/mpi-competency
\CHAPTER{Getting started with MPI}{mpi-started}
\CHAPTER{MPI topic: Functional parallelism}{mpi-functional}
\CHAPTER{MPI topic: Collectives}{mpi-collective}
\CHAPTER{MPI topic: Point-to-point}{mpi-ptp}
\CHAPTER{MPI topic: Communication modes}{mpi-persist}
\CHAPTER{MPI topic: Data types}{mpi-data}
\CHAPTER{MPI topic: Communicators}{mpi-comm}
\CHAPTER{MPI topic: Process management}{mpi-proc}
\CHAPTER{MPI topic: One-sided communication}{mpi-onesided}
\CHAPTER{MPI topic: File I/O}{mpi-io}
\CHAPTER{MPI topic: Topologies}{mpi-topo}
\CHAPTER{MPI topic: Shared memory}{mpi-shared}
\CHAPTER{MPI topic: Tools interface}{mpi-tools}
\CHAPTER{MPI leftover topics}{mpi}
\CHAPTER{MPI Examples}{mpi-examples}
%\CHAPTER{MPI Review}{mpireview}

\acresetall
\part{OpenMP}

\input chapters/omp-competency
\CHAPTER{Getting started with OpenMP}{omp-basics}
\CHAPTER{OpenMP topic: Parallel regions}{omp-parallel}
\CHAPTER{OpenMP topic: Loop parallelism}{omp-loop}
\CHAPTER{OpenMP topic: Reductions}{omp-reduction}
\CHAPTER{OpenMP topic: Work sharing}{omp-share}
\CHAPTER{OpenMP topic: Controlling thread data}{omp-data}
\CHAPTER{OpenMP topic: Synchronization}{omp-sync}
\CHAPTER{OpenMP topic: Tasks}{omp-task}
\CHAPTER{OpenMP topic: Affinity}{omp-affinity}
\CHAPTER{OpenMP topic: Memory model}{omp-memory}
\CHAPTER{OpenMP topic: SIMD processing}{omp-simd}
\CHAPTER{OpenMP topic: Offloading}{omp-gpu}

\CHAPTER{OpenMP remaining topics}{openmp}
%%\CHAPTER{OpenMP Reference}{ompref}
\CHAPTER{OpenMP Review}{ompreview}
\CHAPTER{OpenMP Examples}{omp-examples}

\part{PETSc}

\CHAPTER{PETSc basics}{petsc-design}
\CHAPTER{PETSc objects}{petsc-objects}
\CHAPTER{Grid support}{petsc-dmda}
\CHAPTER{Finite Elements support}{petsc-fem}
\CHAPTER{PETSc solvers}{petsc-solver}
\CHAPTER{PETSC nonlinear solvers}{petsc-nonlinear}
\CHAPTER{PETSc execution on GPUs}{petsc-gpu}
\CHAPTER{PETSc tools}{petsc-tools}
\CHAPTER{PETSc topics}{petsc}

\part{Other programming models}

\lstset{language=Fortran}
\CHAPTER{Co-array Fortran}{caf}
\lstset{language=C++}
\CHAPTER{Sycl, OneAPI, DPC++}{dpcpp}
\CHAPTER{Python multiprocessing}{multiprocessing}

\part{The Rest}

%\CHAPTER{Ruminations on parallelism}{patterns}
\CHAPTER{Exploring computer architecture}{architecture}
%% merged into next \CHAPTER{Process and thread affinity}{affinity}
\CHAPTER{Hybrid computing}{hybrid}
\CHAPTER{Random number generation}{random}
\CHAPTER{Parallel I/O}{io}
\CHAPTER{Support libraries}{libraries}

%% \vfill\pagebreak
%% \appendix
%% \makeatletter
%% \renewcommand\exercisenumber{\Alph{chapter}.\arabic{section}.\arabic{excounter}}
%% \makeatother
%% \setcounter{tocdepth}{1}
%% \addcontentsline{toc}{toc}{Appendices}

%\Level 0 {Theoretical background}

%\input appendices/blurb

%\APPENDIX{Linear algebra}{linear algebra}{norms}

\part{Tutorials}
\label{app:practical}

\input tutorials/blurb

\TUTORIAL{Debugging}{debug} % VLE is this the same as gnudebug in scicompbook?
\TUTORIAL{Tracing, timing, and profiling}{tracing}
\TUTORIAL{SimGrid}{simgrid}
\TUTORIAL{Batch systems}{slurm}
\TUTORIAL{Parallel I/O}{io}

\part{Class projects}

\PROJECT{A Style Guide to Project Submissions}{projectstyle}
\PROJECT{Warmup Exercises}{warmup}
\PROJECT{Mandelbrot set}{mandelbrot}
\PROJECT{Data parallel grids}{grid}
\PROJECT{N-body problems}{nbody}

\part{Didactics}

\CHAPTER{Teaching guide}{mpi-course}
\CHAPTER{Teaching from mental modesl}{mpi-mental}

%\CHAPTER{Introduction to parallel programming}{conwaysection}


\part {Bibliography, index, and list of acronyms}

%% \Level 1 {Ascii table}
%% \input ascii
%% \vfill\pagebreak

\Level 0 {Bibliography}

\bibliography{vle}
\bibliographystyle{plain}
\vfill\pagebreak

\Level 0 {List of acronyms}

\def\acitem#1#2{\item[#1] #2}
\def\acitemi#1#2#3{\item[#1]{#2}\index{#1|see{#3}}}

\begin{multicols}{2}
\begin{description}
\input acronyms
\end{description}
\end{multicols}
\vfill\pagebreak

%\Level 1 {Index}

%Bold reference: defining passage; italic reference: illustration.

\Level 0 {General Index}

\begin{multicols*}{2}
\printindex
\end{multicols*}

\Level 0 {Index of MPI commands and keywords}

\begin{multicols*}{2}
\printindex[mpi]
\end{multicols*}

\Level 1 {From the standard document}

This is an automatically generated list of every
function, type, and constant in the MPI standard document.
Where these appear in this book, a page reference is given.

\Level 2 {List of all functions}
\input{standard/standard-functions}
%% \Level 2 {List of all dtypes}
%% \input{standard/standard-dtypes}
\Level 2 {List of all ctypes}
\begin{multicols}{3}
\catcode`\_=12
\footnotesize
\begin{itemize}
\item \texttt{COMPLEX}~\pageref{def:COMPLEX}
\item \texttt{Datatype}~\pageref{def:Datatype}
\item \texttt{Group}~\pageref{def:Group}
\item \texttt{INTEGER}~\pageref{def:INTEGER}
\item \texttt{LOGICAL}~\pageref{def:LOGICAL}
\item \texttt{MPI_Aint}~\pageref{def:MPI_Aint}
\item \texttt{MPI_Comm_copy_attr_function}~\pageref{def:MPI_Comm_copy_attr_function}
\item \texttt{MPI_Comm_delete_attr_function}~\pageref{def:MPI_Comm_delete_attr_function}
\item \texttt{MPI_Comm_errhandler_fn}~\pageref{def:MPI_Comm_errhandler_fn}
\item \texttt{MPI_Comm_errhandler_function}~\pageref{def:MPI_Comm_errhandler_function}
\item \texttt{MPI_Copy_function}~\pageref{def:MPI_Copy_function}
\item \texttt{MPI_Count}~\pageref{def:MPI_Count}
\item \texttt{MPI_Datarep_conversion_function}~\pageref{def:MPI_Datarep_conversion_function}
\item \texttt{MPI_Datarep_conversion_function_c}~\pageref{def:MPI_Datarep_conversion_function_c}
\item \texttt{MPI_Delete_function}~\pageref{def:MPI_Delete_function}
\item \texttt{MPI_File_errhandler_fn}~\pageref{def:MPI_File_errhandler_fn}
\item \texttt{MPI_File_errhandler_function}~\pageref{def:MPI_File_errhandler_function}
\item \texttt{MPI_Offset}~\pageref{def:MPI_Offset}
\item \texttt{MPI_Session_delete_attr_function}~\pageref{def:MPI_Session_delete_attr_function}
\item \texttt{MPI_Type_copy_attr_function}~\pageref{def:MPI_Type_copy_attr_function}
\item \texttt{MPI_Type_delete_attr_function}~\pageref{def:MPI_Type_delete_attr_function}
\item \texttt{MPI_Win_copy_attr_function}~\pageref{def:MPI_Win_copy_attr_function}
\item \texttt{MPI_Win_delete_attr_function}~\pageref{def:MPI_Win_delete_attr_function}
\item \texttt{MPI_Win_errhandler_fn}~\pageref{def:MPI_Win_errhandler_fn}
\item \texttt{MPI_Win_errhandler_function}~\pageref{def:MPI_Win_errhandler_function}
\item \texttt{REAL}~\pageref{def:REAL}
\item \texttt{Status}~\pageref{def:Status}
\item \texttt{_Bool}~\pageref{def:_Bool}
\item \texttt{bool}~\pageref{def:bool}
\item \texttt{char}~\pageref{def:char}
\item \texttt{class}~\pageref{def:class}
\item \texttt{double}~\pageref{def:double}
\item \texttt{enum}~\pageref{def:enum}
\item \texttt{float}~\pageref{def:float}
\item \texttt{int}~\pageref{def:int}
\item \texttt{long}~\pageref{def:long}
\item \texttt{short}~\pageref{def:short}
\item \texttt{wchar_t}~\pageref{def:wchar_t}
\end{itemize}
\end{multicols}

\Level 2 {List of all ftypes}
\input{standard/standard-ftypes}
\Level 2 {List of all constants}
\input{standard/standard-consts}
\Level 2 {List of all callbacks}
\input{standard/standard-callbacks}

\Level 1 {MPL commands and topics}
\label{sec:idx:mpl}

\begin{multicols*}{2}
  \printindex[mpl]
\end{multicols*}

\Level 1 {Python}

\Level 2 {Notes}

\begin{multicols*}{2}
  \printindex[python]
\end{multicols*}

\Level 2 {Buffer specifications}

{\small
%\begin{multicols}{2}
  \verbatiminput{standardp/bufspec.tex}
  \verbatiminput{standardp/bufspecb.tex}
  \verbatiminput{standardp/bufspecv.tex}
  \verbatiminput{standardp/bufspecw.tex}
%\end{multicols}
}

\Level 2 {Listing of python routines}

\begin{multicols}{3}
\small
\pylist{Comm}
\pylist{Cartcomm}
\pylist{Distgraphcomm}
\pylist{Graphcomm}
\pylist{Intercomm}
\pylist{Intracomm}
\pylist{Topocomm}
\pylist{Group}

\pylist{Request}
\pylist{Grequest}
\pylist{Prequest}
\pylist{Status}

\pylist{Win}

\pylist{Datatype}
\pylist{File}
\pylist{Info}
\pylist{Op}

\pylist{Errhandler}
\pylist{Message}
%\pylist{_InPlace}
%\pylist{Exception}
%\pylist{_Pickle}
\end{multicols}

\Level 0 {Index of OpenMP commands}

\begin{multicols*}{2}
\printindex[omp]
\end{multicols*}

\Level 0 {Index of PETSc commands}

\begin{multicols*}{2}
\printindex[petsc]
\end{multicols*}

\Level 0 {Index of SYCL commands}

\begin{multicols*}{2}
\printindex[sycl]
\end{multicols*}

\hbox{}\vfill
\includegraphics{isbnbarcode}

\closeout\chapterlist
\end{document}
