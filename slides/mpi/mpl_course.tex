% -*- latex -*-
%%%%%%%%%%%%%%%%%%%%%%%%%%%%%%%%%%%%%%%%%%%%%%%%%%%%%%%%%%%%%%%%
%%%%%%%%%%%%%%%%%%%%%%%%%%%%%%%%%%%%%%%%%%%%%%%%%%%%%%%%%%%%%%%%
%%%%
%%%% This text file is part of the source of 
%%%% `Parallel Computing'
%%%% by Victor Eijkhout, copyright 2012-2020
%%%%
%%%% mpl_course.tex : course in MPL interface to MPI
%%%%
%%%%%%%%%%%%%%%%%%%%%%%%%%%%%%%%%%%%%%%%%%%%%%%%%%%%%%%%%%%%%%%%
%%%%%%%%%%%%%%%%%%%%%%%%%%%%%%%%%%%%%%%%%%%%%%%%%%%%%%%%%%%%%%%%

\documentclass[11pt,headernav]{beamer}

\beamertemplatenavigationsymbolsempty
\usetheme{Madrid}%{Montpellier}
\usecolortheme{seahorse}
\setcounter{tocdepth}{1}
%% \AtBeginSection[]
%% {
%%   \begin{frame}
%%     \frametitle{Table of Contents}
%%     \tableofcontents[currentsection]
%%   \end{frame}
%% }

\setbeamertemplate{footline}{\hskip1em Eijkhout: MPI course\hfill
  \hbox to 0in {\hss \includegraphics[scale=.1]{tacclogonew}}%
  \hbox to 0in {\hss \arabic{page}\hskip 1in}}

\usepackage{multicol,multirow}
% custom arrays and tables
\usepackage{array} %,multirow,multicol}
\newcolumntype{R}{>{\hbox to 1.2em\bgroup\hss}{r}<{\egroup}}
\newcolumntype{T}{>{\hbox to 8em\bgroup}{c}<{\hss\egroup}}

\input acromacs
\input slidemacs
\input coursemacs
\input listingmacs

\includecomment{full}
\excludecomment{condensed}
\excludecomment{online}

\specialcomment{tacc}{\stepcounter{tacc}\def\CommentCutFile{tacc\arabic{tacc}.cut}}{}
\newcounter{tacc}
%\excludecomment{tacc}
\includecomment{xsede}
\includecomment{utonly}

\includecomment{onesided}
\includecomment{advanced}
\includecomment{foundations}

\def\Location{}% redefine in the inex file
\def\courseyear{2020}
\def\Location{TACC APP institute MPI training \courseyear}
\begin{xsede}
  \def\Location{TACC/XSEDE MPI training \courseyear}
\end{xsede}
\def\TitleExtra{}

%%%%%%%%%%%%%%%%
%%%%%%%%%%%%%%%% Document
%%%%%%%%%%%%%%%%

\begin{document}
\parskip=10pt plus 5pt minus 3pt

\title{Tutorial on the MPL interface to MPI}
\author{Victor Eijkhout {\tt eijkhout@tacc.utexas.edu}}
\date{\Location}

\begin{frame}
  \titlepage
\end{frame}

\begin{xsede}
  \input xsede-conduct
\end{xsede}
%% \begin{utonly}
%%   \input tacc-conduct
%% \end{utonly}

\begin{frame}{Justification}
  While the C API to MPI is usable from C++, it feels very unidiomatic
  for that language.
  MPL is a modern C++11 interface to MPI.
  It is both idiomatic and elegant, simplifying many calling sequences.
\end{frame}

\Level 0 {Basics}

\begin{frame}[containsverbatim]{Header file}
  \input{mplnote-header-file.cut}
\end{frame}
\begin{frame}[containsverbatim]{Init, finalize}
  \input{mplnote-init,-finalize.cut}
\end{frame}
\begin{frame}[containsverbatim]{World communicator}
  \input{mplnote-world-communicator.cut}
\end{frame}
\begin{frame}[containsverbatim]{Processor name}
  \input{mplnote-processor-name.cut}
\end{frame}
\begin{frame}[containsverbatim]{Rank and size}
  \input{mplnote-rank-and-size.cut}
\end{frame}
\begin{frame}[containsverbatim]{Timing}
  \input{mplnote-timing.cut}
\end{frame}
\begin{frame}[containsverbatim]{Predefined communicators}
  \input{mplnote-predefined-communicators.cut}
\end{frame}

\Level 0 {Collectives}

\begin{frame}[containsverbatim]{Scalar buffers}
  \input{mplnote-scalar-buffers.cut}
\end{frame}
\begin{frame}[containsverbatim]{Vector buffers}
  \input{mplnote-vector-buffers.cut}
\end{frame}
\begin{frame}[containsverbatim]{Iterator buffers}
  \input{mplnote-iterator-buffers.cut}
\end{frame}

\begin{frame}[containsverbatim]{Gather scatter}
  \input{mplnote-gather-scatter.cut}
\end{frame}
\begin{frame}[containsverbatim]{Broadcast}
  \input{mplnote-broadcast.cut}
\end{frame}
\begin{frame}[containsverbatim]{Reduce in place}
  \footnotesize
  \input{mplnote-reduce-in-place.cut}
\end{frame}
\begin{frame}[containsverbatim]{Reduction operator}
  \input{mplnote-reduction-operator.cut}
\end{frame}
\begin{frame}[containsverbatim]{User defined operators}
  \input{mplnote-user-defined-operators.cut}
\end{frame}
\begin{frame}[containsverbatim]{Non-blocking collectives}
  \input{mplnote-non-blocking-collectives.cut}
\end{frame}

\Level 0 {Point-to-point communication}

\begin{frame}[containsverbatim]{Blocking send and receive}
  \input{mplnote-blocking-send-and-receive.cut}
\end{frame}
\begin{frame}[containsverbatim]{Sending arrays}
  \input{mplnote-sending-arrays.cut}
\end{frame}
\begin{frame}[containsverbatim]{Message tag}
  \input{mplnote-message-tag.cut}
\end{frame}
\begin{frame}[containsverbatim]{Any source}
  \input{mplnote-any-source.cut}
\end{frame}
\begin{frame}[containsverbatim]{Status object}
  \input{mplnote-status-object.cut}
\end{frame}
\begin{frame}[containsverbatim]{Send-recv call}
  \input{mplnote-send-recv-call.cut}
\end{frame}
\begin{frame}[containsverbatim]{Status querying}
  \input{mplnote-status-querying.cut}
\end{frame}

\begin{frame}[containsverbatim]{Receive count}
  \input{mplnote-receive-count.cut}
\end{frame}

\begin{frame}[containsverbatim]{Requests from non-blocking calls}
  \input{mplnote-requests-from-non-blocking-calls.cut}
\end{frame}
\begin{frame}[containsverbatim]{Wait any}
  \input{mplnote-wait-any.cut}
\end{frame}

\begin{frame}[containsverbatim]{Buffered send}
  \input{mplnote-buffered-send.cut}
\end{frame}
\begin{frame}[containsverbatim]{Buffer attach and detach}
  \input{mplnote-buffer-attach-and-detach.cut}
\end{frame}
\begin{frame}[containsverbatim]{Persistent requests}
  \input{mplnote-persistent-requests.cut}
\end{frame}

\Level 0 {Derived Datatypes}

\begin{frame}[containsverbatim]{Datatypes}
  \input{mplnote-data-types.cut}
\end{frame}
\begin{frame}[containsverbatim]{Derived type handling}
  \input{mplnote-derived-type-handling.cut}
\end{frame}
\begin{frame}[containsverbatim]{Contiguous type}
  \input{mplnote-contiguous-type.cut}
\end{frame}
\begin{frame}[containsverbatim]{Vector type}
  \input{mplnote-vector-type.cut}
\end{frame}
\begin{frame}[containsverbatim]{Subarray layout}
  \input{mplnote-subarray-layout.cut}
\end{frame}
\begin{frame}[containsverbatim]{Indexed type}
  \input{mplnote-indexed-type.cut}
\end{frame}
\begin{frame}[containsverbatim]{Struct type}
  \input{mplnote-struct-type.cut}
\end{frame}

\Level 0 {Communicator manipulations}

\begin{frame}[containsverbatim]{Split by shared memory}
  \input{mplnote-split-by-shared-memory.cut}
\end{frame}

\begin{frame}[containsverbatim]{Communicator errhandler}
  \input{mplnote-communicator-errhandler.cut}
\end{frame}
\begin{frame}[containsverbatim]{Communicator splitting}
  \input{mplnote-communicator-splitting.cut}
\end{frame}

\Level 0 {Process topologies}

\Level 0 {Other}

\begin{frame}[containsverbatim]{Timing}
  \input{mplnote-timing.cut}
\end{frame}

\begin{frame}[containsverbatim]{Threading support}
  \input{mplnote-threading-support.cut}
\end{frame}

\begin{frame}{Missing material}
  \begin{itemize}
  \item File I/O
  \item One-sided communication
  \item Shared memory
  \item Process management
  \end{itemize}
\end{frame}

\begin{comment}
  \Level 0 {MPI File I/O}
  \Level 0 {One-sided communication}
  \Level 0 {Shared memory}
  \Level 0 {Process management}
\end{comment}

\end{document}

